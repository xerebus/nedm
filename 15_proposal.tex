\documentclass[twocolumn,aps,prb,citeautoscript]{revtex4-1}
\usepackage{graphicx,amsmath,amssymb,subcaption,tabularx,array}

\begin{document}

\newcolumntype{R}[1]{>{\raggedleft\arraybackslash}p{#1}}

\title{Proposal: Magnetic field profile of a third-scale model for the nEDM experiment}
\author{A. Biswas}
\affiliation{Kellogg Radiation
Laboratory, California Institute of Technology}

%\begin{abstract}
%Discovery of a non-zero electric dipole moment in the neutron (nEDM) would indicate a CP violation,
%with implications for
%extending the Standard Model and confirming predictions about matter-antimatter asymmetry.
%Experiments using shifts in neutron
%precession frequency to measure the nEDM require a uniform magnetic field to prevent false signals.
%\end{abstract}

\maketitle

\section{Motivation}

Charge-parity (CP) violation plays a critical role in explaining matter-antimatter asymmetry in the universe
and guiding theories beyond the Standard Model.
The experimental search for instances of CP violation is ongoing \cite{cpv,ill}.
A nonzero electric dipole moment (EDM) in the neutron would constitute such a violation and can be measured
through the shift in Larmor precession frequency of
ultra-cold neutrons (UCN) in electric and magnetic fields \cite{pendlebury}.
An upper limit of $2.9 \times 10^{-26}\;e\cdot\text{cm}$ was established in 2006\cite{ill}.

Experiments using these methods 
are prone to a geometric phase (GP) effect, caused by gradients of the magnetic
field, that can lead to a false signal: a nonzero measurement even if the true EDM is zero
\cite{ill,pendlebury}. Eliminating the GP effect requires precise engineering
to create a space-uniform magnetic field. Field uniformity also elongates the neutron spin polarization
lifetime \cite{coil}.

Here, we describe a third-scale model of a shielded magnet
designed to create a field suitable for a more precise nEDM measurement.
New structural features of the third-scale are based on studies of a previous
half-scale model and simulations of superconducting shielding geometries. 

Precise measurements of the magnetic field profile in this new model are needed to determine the potential
effectiveness of its components in the full-scale nEDM experiment, which will take place at Oak Ridge
National Laboratory (ORNL). Field gradients near the shielding and near
the center of the magnet are particularly good indicators of how well the
superconducting shielding can enforce field uniformity and mitigate the GP effect.

\section{Previous work}

%The third-scale model was inspired by 2013-14
%experiments on the field profile and shielding of a half-scale model.

Over the summer of 2014, I studied
the effectiveness of a specific component of the previous nEDM magnet model:
a superconducting lead endcap designed to
mitigate edge effects in the magnetic field. Measured field maps with the
closely matched simulations and indicated that the endcap shifts field peaks
away from magnet center, decreasing field gradients in the desired regions.
My studies concluded that the presence of a single endcap in the previous model
effectiely mitigated edge effects.

This result justifies the new design in the third-scale model: a shorter cylinder where the ends are much
closer to the measurement cells. Two endcaps are included in the third-scale to control edge effects. The
resulting design is much more compact.

%Shorter and smaller magnet models were considered troublesome because the edge effects would occur close to
%the measurement cells (which were placed at the center), so field uniformity in the cells would be disturbed.
%This was particularly undesirable because shorter magnet models have practical advantages for the full-scale
%experiment at ORNL, such as the ability to keep all cryostat components upright during the experiment.
%
%Confirming that we could mitigate edge effects with superconducting
%endcaps made shorter models viable, and thus the third-scale model
%- a shorter, fatter cylinder than the half-scale model - was proposed.

\section{Constructing the 1/3-scale model}

The construction process for third-scale model involves simulations of slight variations in the design,
finalizing the design, building new components (both the magnetic coil and shielding),
and assembly in the synchrotron lab in Lauritsen. I started work on the simulations in December
and will be involved in the construction process during the school year.

Structurally, the third-scale model is composed of various cylindrical components nested inside one another
in a Russian doll type structure.
The $B_0$ coil, a cylindrical structure with wires coiled along its surface, produces a magnetic field pointing
primarily in a direction orthogonal to the cylinder's central axis. The wires are arranged in a particular
$\cos\theta$ coil geometry to produce this field \cite{coil}. An initial version of the $B_0$ coil for the third-scale
model will be adapted from the one used in the previous model. Studies on the effect of the coil's distortion
parameter, a measure of the wire geometry, on field uniformity will guide the design of an updated version.

Surrounding the $B_0$ coil are several open-ended cylindrical shields.
A ferromagnetic Metglas shield promotes field uniformity
by reinforcing field lines perpendicular to its surface. Outside the Metglas, there is an axial lead shield which
goes superconducting when the magnet is placed in a cryostat and cooled below 7 K. This shield protects the
field inside the magnet from environmental fluctuations by redirecting external field lines along its surface.

Two lead endcaps close the top and bottom ends of the axial lead shield. The top endcap contains a small
central hole to allow
a magnetic probe to enter the cavity inside the $B_0$ coil, while the bottom endcap is circular. As noted,
previous studies suggest that these endcaps will mitigate edge effects and enforce field uniformity near
magnet center; the measurements I plan to take this summer will determine their effectiveness. The model is encased
in an aluminum shell for uniform cooling inside the cryostat.

\section{Methods \& challenges}

The apparatus used for magnetic field measurements on the third-scale model, such as the magnetic probe,
mounts, etc. will be determined as the third-scale design is finalized over February - April.

Over the summer, magnetic field measurements will be taken with a probe that will enter the $B_0$ cavity
through the central opening in the top lead endcap. The size of this opening is small to make the endcap
more effective, but this presents a challenge in moving the probe.
Though the mounting mechanism for the probe is yet to be finalized, we will most likely use an
elbow-style mount that allows the probe to move more freely inside the $B_0$ cavity despite the small opening.

This new mount, and the overall new structure of the third-scale model,
will introduce various new systematic errors to identify and correct before comparing our measured field maps
with simulations. This will require significant adaptation of
the data analysis code written for the previous model: conversion from the mount's geometry to
Cartesian coordinates, interfacing with the new probe, and more.
The variable angle of the mount will also
introduce new uncertainties which will need to be accurately determined by experiment.

\bigskip
\bigskip
\bigskip

\section{Timeline}

The following is a list of checkpoints with approximate expected times to completion. Some of these tasks
are disjoint. For example: assembly of the third-scale model will begin during the year; the expected time shown
in the period of work that will fall during SURF.
Fata acquisition will happen in multiple stages depending on our cryostat cooldown
schedule, and work on other tasks will be done between these stages.

\begin{table}[h!]
    \begin{tabular}{l|r}
        Task & Time \\\hline\hline
        Assemble third-scale model & 2 weeks \\
        Develop LabView interfaces for new components & 1 week \\
        Adapt 2014 data analysis code for new interfaces & 2 weeks \\
        Data acquisition, room temp. and superconducting & 2 weeks \\
        Systematic error analysis and correction & 3 weeks
    \end{tabular}
\end{table}

\begin{thebibliography}{}
\bibitem{cpv} Cronin, J. ``Nobel Lecture: CP Symmetry Violation – The Search
for Its Origin,'' Nobel Media AB (2013).
\bibitem{ill} Baker, C. A., D. D. Doyle, P. Geltenbort, K. Green, M. G. D. Van der Grinten, P. G. Harris, P. Iaydjiev et al. ``Improved experimental limit on the electric dipole moment of the neutron.'' \textit{Physical Review Letters} 97, no. 13 (2006): 131801.
\bibitem{pendlebury} Pendlebury et. al. ``Geometric-phase-induced false electric dipole moment signals for particles in traps.'' \textit{Phys. Rev. A.} 70, 032102 (2004).
\bibitem{krl} ``Search for the nEDM at Caltech.'' Kellogg Radiation Laboratory
$\langle$krl.caltech.edu$\rangle$ (2014).
\bibitem{endcapstyles} Malkowski, S., R. Y. Adhikari, J. Boissevain, C. Daurer, B. W. Filippone, B. Hona, B. Plaster, D. Woods, and H. Yan. ``Overlap Technique for End-Cap Seals on Cylindrical Magnetic Shields.'' \textit{IEEE Transactions on Magnetics} 49, no. 1 (2013): 651-653.
\bibitem{coil} Perez Galvan, A., B. Plaster, J. Boissevain, R. Carr, B. W. Filippone, M. P. Mendenhall, R. Schmid, R. Alarcon, and S. Balascuta. ``High uniformity magnetic coil for search of neutron electric dipole moment.'' \textit{Nuclear Instruments and Methods in Physics Research Section A: Accelerators, Spectrometers, Detectors and Associated Equipment} 660, no. 1 (2011): 147-153.
\bibitem{rotshield} Mendenhall, M. P. \texttt{RotationShield} source. $\langle$https://github.com/mpmendenhall/rotationshield$\rangle$ (2014).
\end{thebibliography}

\end{document}
