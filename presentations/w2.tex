\documentclass{beamer}
\usepackage{graphicx}

\newcommand{\pyplot}{\includegraphics[width=\textwidth, trim=60px 60px 60px 40px]}


\begin{document}

\title{Week 2}
\maketitle

\begin{frame}
\frametitle{\texttt{plotter} script} \pause

    \begin{itemize}
    \item 324 sloc \pause
    \item reads \texttt{RotationShield} and FieldMapping VI input \pause
    \item uses new normalization method \pause
        \begin{itemize}
        \item average of data points near $(0, 0, 0)$ vs. polynomial fit \pause
        \item calculates desired normalization level - average $B_x$ of measured maps \pause
        \end{itemize}
    \item handles custom field slices \pause
    \item to-do: field gradients, interpolation, smooth plots
    \end{itemize}

\end{frame}


\begin{frame}
\frametitle{10 cm offset between map \& simulation}

    \begin{center}
    \pyplot{../savedplots/original_Bz.eps}
    \end{center}

\end{frame}

\begin{frame}
\frametitle{10 cm offset between map \& simulation}

    \begin{center}
    \pyplot{../savedplots/original_Bx.eps}
    \end{center}

\end{frame}

\begin{frame}
\frametitle{modifications to account for 10 cm offset}

    \begin{itemize}
    \item varied metglas thickness from 5 cm to 1 mm \\ (closest to actual) \pause
    \item extended metglas slightly (2 cm) above $B_0$ coil \pause
    \item extended metglas far (10 cm) above $B_0$ coil to highlight effects
    \end{itemize}

\end{frame}


\begin{frame}
\frametitle{varying thickness: small change in $B$ magnitude, no shift}

    \begin{center}
    \pyplot{../savedplots/thickness_Bz.eps}
    \end{center}

\end{frame}

\begin{frame}
\frametitle{varying thickness: small change in $B$ magnitude, no shift}

    \begin{center}
    \pyplot{../savedplots/thickness_Bx.eps}
    \end{center}

\end{frame}


\begin{frame}
\frametitle{2 cm longer on top: small magnitude change}

    \begin{center}
    \pyplot{../savedplots/longer2cm_Bz.eps}
    \end{center}

\end{frame}

\begin{frame}
\frametitle{2 cm longer on top: small magnitude change}

    \begin{center}
    \pyplot{../savedplots/longer2cm_Bx.eps}
    \end{center}

\end{frame}

\begin{frame}
\frametitle{10 cm longer on top: confirms magnitude change}

    \begin{center}
    \pyplot{../savedplots/longer10cm_Bz.eps}
    \end{center}

\end{frame}

\begin{frame}
\frametitle{10 cm longer on top: confirms magnitude change}

    \begin{center}
    \pyplot{../savedplots/longer10cm_Bx.eps}
    \end{center}

\end{frame}

\begin{frame}
\frametitle{other things to try} \pause

    \begin{itemize}
    \item metglas shearing - hard to model \pause
        \begin{itemize}
        \item \texttt{RotationShield} handles arbitrary fields (e.g. you can put \\ a line
        current anywhere) \pause
        \item but before applying the field it builds an interaction matrix - \\ requires
        azimuthal symmetry \pause
        \end{itemize}
    \item rigorously check centering, dimensions of experimental setup \pause
    \end{itemize}

\end{frame}

\end{document}
