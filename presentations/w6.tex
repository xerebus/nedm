\documentclass{beamer}
\usepackage{graphicx, amsmath, amssymb, textcomp}

\newcommand{\pyplot}{\includegraphics[width=\textwidth, trim=60px 60px 60px 40px]}


\begin{document}

\title{Week 6: Centering and Axial Shield Effects}
\maketitle

\begin{frame}
\frametitle{current data analysis steps}

    \begin{enumerate}
    \item subtract backgrounds
    \item correct for probe mis-centering along $x$ axis
    \item correct for $B_x$, $B_y$, $B_z$ probe separations
    \item estimate probe tilt angle and correct $B_z$
    \item normalize simulated maps to $B_z$ at magnet center
    \end{enumerate}

    \begin{itemize}
    \item step 2 ($x$ centering) will be done differently
    \item steps 3 \& 4 are new this week
    \end{itemize}

\end{frame}


\begin{frame}
\frametitle{correction 2: $x$ centering}

    \begin{enumerate}
    \item plot $B_z$ v. $z$ along $x = 0, y = 0$
        \begin{itemize}
        \item if probe is perfectly centered, curve will be flat
        \item height of actual curve peak suggests how far off-center we are
        \end{itemize}
    \item guess $x$ offsets
    \item plot $B_z$ v. $z$ along extremities $x = \pm 0.1 \text{ m}, y = 0$
    \end{enumerate}

    \begin{itemize}
    \item doing step 1 for three maps:
        \begin{itemize}
            \item axial shield normal, endcap normal (9:43/9:27)
            \item axial shield SC, endcap unknown (12:35/12:20)
            \item axial shield SC, endcap SC (13:38/13:24)
        \end{itemize}
    \end{itemize}

\end{frame}

\begin{frame}
\frametitle{$B_z$ v. $z$ along $x = 0, y = 0$ for three maps}

    \begin{center}
    \pyplot{../savedplots/082014/x_offset_difference.eps}
    \end{center}

\end{frame}

\begin{frame}
\frametitle{correction 2: $x$ centering (cont.)}

    \begin{itemize}
    \item peak heights are substantially different
    \item two possibilities:
        \begin{itemize}
        \item we are seeing a small-scale version of the effect of the endcap
        \item or, probe became further offset in $x$ between 12:35 and 13:38
        \end{itemize}
    \item probably a mix of both
    \item best to use 4mm $x$ offset for all three maps - assuming that the probe became
    further offset between the normal and SC measurements would probably be confirmation bias
    \end{itemize}

\end{frame}

\begin{frame}
\frametitle{(axial normal, endcap normal) and \\(axial SC, endcap unknown) agree}

    \begin{center}
    \pyplot{../savedplots/082014/normal_axial_data_agree.eps}
    \end{center}

\end{frame}

\begin{frame}
\frametitle{both agree with Metglas-only simulation \\(no axial, no endcap)}

    \begin{center}
    \pyplot{../savedplots/082014/normal_axial_agree.eps}
    \end{center}

\end{frame}

\begin{frame}
\frametitle{(axial normal, endcap normal) and \\(axial SC, endcap unknown)}

    \begin{itemize}
    \item agrees with Simon's plots
    \item when endcap state was unknown, it was likely not SC
    \item effect of the axial shield alone going SC seems to be minimal \\(as expected by design)
    \bigskip
    \item but simulation predicts stronger effect from axial shield!
    \item this effect is highly dependent on axial shield length
    \end{itemize}

\end{frame}

\begin{frame}
\frametitle{simulations: axial shield not present, and present with lengths {1100, 1105, 1110} mm}

    \begin{center}
    \pyplot{../savedplots/082014/axial_effect.eps}
    \end{center}

\end{frame}

\begin{frame}
\frametitle{simulations: same, but with SC endcap at $z = 1.128$ m\\ in all four cases}

    \begin{center}
    \pyplot{../savedplots/082014/axial_effect_endcap.eps}
    \end{center}

\end{frame}

\begin{frame}
\frametitle{axial shield effects}

    \begin{itemize}
    \item we expect axial shield effect to be negligible
    \item effect is indeed negligible when axial shield is 1100 mm tall
    \item but taller axial shields have a substantial $B_z$ suppression effect
    \item presence of SC endcap hides this suppression effect
    \end{itemize}

    \begin{itemize}
    \item 1100 mm is closest to Metglas height (1080 mm)
    \item perhaps what really matters is the height difference between the Metglas and axial shields?
    \item next: simulations with axial shield height fixed at 1110 mm, metglas height varying $\pm 5$ mm
    \end{itemize}

\end{frame}

\begin{frame}
\frametitle{axial shield and Metglas height difference}

    \begin{center}
    \pyplot{../savedplots/082014/axial_effect_metglas.eps}
    \end{center}

\end{frame}

\begin{frame}
\frametitle{thoughts}

    \begin{itemize}
    \item axial shield's $B_z$ suppression effect correlated with axial-Metglas height difference:\\ 
    the further the axial shield extends above the Metglas, the more it suppresses $B_z$ (lower
    peak height)
    \item SC endcap hides this suppression effect
        \begin{itemize}
        \item suppression effect desirable?
        \item simulate extending the axial shield in lieu of the endcap to create better field uniformity?
        \end{itemize}
    \end{itemize}

\end{frame}

\begin{frame}
\frametitle{thoughts (cont.)}

    \begin{itemize}
    \item how do we explain the agreement between (axial normal, endcap normal) and (axial SC, endcap unknown)
    in the data?
    \end{itemize}
    \begin{enumerate}
    \item maybe the real axial-Metglas height difference is smaller than measured? 
        \begin{itemize}
        \item we measured the heights of the axial and Metglas shields individually
        \item perhaps reinforce this by measuring the height difference and gap - $z$ centering may be off
        \item if the gap is very small, then the axial shield suppression effect will be negligible (as shown
        in the simulation with axial shield height at 1100 mm)
        \end{itemize}
    \item maybe the endcap was partially SC, somehow enough to hide an axial shield suppression effect, but
    not enough to match the (axial SC, endcap SC) simulation? (very unlikely)
    \end{enumerate}

\end{frame}

\end{document}
